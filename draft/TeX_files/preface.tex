\chapter*{Preface}
What a time it is to be a 3D artist!  The tools, simulators, rendering packages, modeling software, and hardware all give us what we need to make stunning images and animations.  To add to that we have many free tools at our disposal and thus lowering the barrier to entry to new students and artists coming from other disciplines.

This book aims at crossing the bridge between The \mbox{Blender Foundation's} Blender\footnote{\url{http://www.blender.org}} and Pixar's RenderMan\footnote{\url{http://renderman.pixar.com}} renderer (for the rest of the book let's refer to RenderMan as \emph{PRMan}).  Both are released for free and  Blender has no restrictions on making money from anything you made with it.   PRMan \emph{does} requires a commercial license if you plan on selling your work directly.  For more information about monetization of your work visit Pixar's RenderMan web site for the legal details.  But for making images for fun, to build a portfolio, or for research it is a great piece of software to use as you get the full package (minus Tractor)

This, however, is introduction to neither Blender or PRMan.  Rather, this book explores the common grounds between the two software packages and gives you the logical ``glue'' to connect the two.

\section*{How to Read This Book}
Throughout the book you will see items in a \textsf{Sans-Serif} font that denotes buttons, labels, or other on-screen type found in Blender, it, or LocalQueue.
% At a later date I would like to have asides in the margins where the more esoteric tidbits can be moved so that the flow is maintained.
Links such as "\url{http://renderman.pixar.com}" appear as dark blue text and can lead you to either web pages or locations within the book itself.  (Assuming you aren't reading this on paper!)


